\documentclass{article}
\usepackage{graphicx} % Required for inserting images
\usepackage{color}
\usepackage{cite}

\title{Lab02-ED-G02-JUAN CAMILO AGUIRRE GONZALEZ-200156480}
\author{FUNDACIÓN UNIVERSIDAD DEL NORTE }
\date{October 23 de 2023}

\begin{document}

\maketitle

\section{Latex}
\subsection{Formulas matematicas}
Utilice su ambiente matematico preferido para escribrir la siguiente formula:

$$f(x)=\sum_{i=1}^{n-1}[100(x_{i+1}-x_{i}^2)^2+(1-x_{i})^2]$$ donde   $$X = (x_{1},...,x_{n})\epsilon  R^n. $$

\section{Tablas}

Reproduzca por lo menos cuatro lineas de la siguiente tabla:

\vspace{0.5cm}



    \begin{tabular}{c|c|c|c|c}
         Bits &  Cuántos números & Cuales números & Mínimo & Máximo\\
         1 & 2^1 & {0, 1} & 0 & 2^{1-1} \\
         2 & 2^2 & {00,01, 10, 11}  & 0 & 2^2-1 \\
         3 & 2^3 & {000,001, 010, 011, 100, 101,110,111} & 0 & 2^3-1  \\
         ... & ... & ... & ... & ... \\
         n & 2^n & {000...00, 000...01, ... 111 ... 11} & 0  & 2^{n-1} \\  
    \end{tabular}

    \hspace{4.5cm}\caption{Table 1:Caption}
    \label{tab:my_label}



    

\newpage
\section{Creación de nuevos comandos}
Crear un comando(nombre tc) que haga lo siguiente:
• Espacio en blanco

• Enter

• Texto(argumento 2) a color(argumento 1) en negrilla (color por defecto
azul)

• Espacio en blanco

• Enter

\newcommand{\tc}[2][\color{blue}]{ \\ \textbf{#2} #1 \\}  

\vspace{0.5cm}

\tc{argumento#1}{argumento#2}



\color{black}
\section{Citas, o referencias, bibliograficas}

Busque las siguientes citas, en cualquier formato(APA, MLA, CHICAGO, etc.),
e incluyalas en el documento como Bibliografıa:

• A Posterior Ensemble Kalman Filter Based On A Modified Cholesky Decomposition

• A Maximum Likelihood Ensemble Filter via a Modified Cholesky Decomposition for Non-Gaussian Data Assimilation

• Diseño y construccion de algoritmos

\vspace{0.6cm}

hola profesor \cite{a39d2ed1-0f9a-3d52-a4b8-ccd73abfd907}
espero se encuentre bien \cite{atmos8070125}
el dia de hoy \cite{s20030877}

\bibliography{bib/bibtestLAB}
\bibliographystyle{plain}

\end{document}
